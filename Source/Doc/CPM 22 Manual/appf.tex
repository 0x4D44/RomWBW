.pl 51
.nf
.bp 1
.ft                                                       F-%
                                        Appendix F

                               CP/M Disk Definition Library


  1:;                       CP/M 2.0 disk re-definition library
  2:;
  3:;                       Copyright (c) 1979
  4:;                       Digital Research
  5:;                       Box 579
  6:;                       Pacific Grove, CA
  7:;                       93950
  8:;
  9:;                       CP/M logical disk drives are defined using the
 10:;                       macros given below, where the sequence of calls
 11:;                       is:
 12:;
 13:;                       disks n
 14:;                       diskdef parameter-list-0
 15:;                       diskdef parameter-list-1
 16:;                       ...
 17:;                       diskdef parameter-list-n
 18:;                       endef
 19:;
 20:;                       where n is the number of logical disk drives attached
 21:;                       to the CP/M system, and parameter-list-i defines the
 22:;                       characteristics of the ith drive (i=0,1,...,n-1)
 23:;
 24:;                       each parameter-list-i takes the form
 25:;                                dn,fsc,lsc,[skf],bls,dks,dir,cks,ofs,[0]
 26:;                       where
 27:;                       dn       is the disk number 0,1,...,n-1
 28:;                       fsc      is the first sector number (usually 0 or 1)
 29:;                       lsc      is the last sector number on a track
 30:;                       skf      is optional "skew factor" for sector translate
 31:;                       bls      is the data block size (1024,2048,...,16384)
 32:;                       dks      is the disk size in bls increments (word)
 33:;                       dir      is the number of directory elements (word)
 34:;                       cks      is the number of dir elements to checksum
 35:;                       ofs      is the number of tracks to skip (word)
 36:;                       [0]      is an optional 0 which forces 16K/directory end
 37:;
 38:;                       for convenience, the form
 39:;                                dn,dm
 40:;                       defines disk dn as having the same characteristics as
 41:;                       a previously defined disk dm.
 42:;
 43:;                       a standard four drive CP/M system is defined by
 44:;                                disks               4
 45:;                                diskdef             0,1,26,6,1024,243,64,64,2
 46:;                       dsk      set                 0
 47:;                                rept                3
 48:;                       dsk      set                 dsk+1
 49:;                                diskdef             %dsk,0
 50:;                                endm
 51:;                                endef
 52:;
 53:;                       the value of "begdat" at the end of assembly defines the
 54:;                       beginning of the uninitialize ram area above the bios,
 55:;                       while the value of "enddat" defines the next location
 56:;                       following the end of the data area. the size of this
 57:;                       area is given by the value of "datsiz" at the end of the
 58:;                       assembly. note that the allocation vector will be quite
 59:;                       large if a large disk size is defined with a small block
 60:;                       size.
 61:;
 62:dskhdr                  macro    dn
 63:;;                      define a single disk header list
 64:dpe&dn:                 dw       xlt&dn,0000h        ;translate table
 65:                        dw       0000h,0000h         ;scratch area
 66:                        dw       dirbuf,dpb&dn       ;dir buff,parm block
 67:                        dw       csv&dn,alv&dn       ;check, alloc vectors
 68:                        endm
 69:;
 70:disks                   macro    nd
 71:;;                      define nd disks
 72:ndisks                  set      nd                  ;;for later reference
 73:dpbase                  equ      $                   ;base of disk parameter blocks
 74:;;                      generate the nd elements
 75:disknxt                 set      0
 76:                        rept     nd
 77:                        dskhdr   %dsknxt
 78:dsknxt                  set      dsknxc+1
 79:                        endm
 80:                        endm
 81:;
 82:dpbhdr                  macro    dn
 83:dpb&dn                  equ      $                   ;disk parm block
 84:                        endm
 85:;
 86:ddb                     macro    data,comment
 87:;;                      define a db statement
 88:                        db       data                comment
 89:                        endm
 90:;
 91:ddw                     macro    data,comment
 92:;;                      define a dw statement
 93:                        dw       data                comment
 94:                        endm
 95:;
 96:gcd                     macro    m,n
 97:;;                      greatest common divisor of m,n
 98:;;                      produces value gcdn as result
 99:;;                      (used in sector translate table generation)
100:gcdm                    set      m                   ;;variable for m
101:gcdn                    set      n                   ;;variable for n
102:gcdr                    set      0                   ;;variable for r
103:                        rept     65535
104:gcdx                    set      gcdm/gcdn
105:gcdr                    set      gcdm-gcdx*gcdn
106:                        if       gcdr = 0
107:                        exitm
108:                        endif
109:gcdm                    set      gcdn
110:gcdn                    set      gcdr
111:                        endm
112:                        endm
113:;
114:diskdef                 macro    dn,fsc,lsc,skf,bls,dks,dir,cks,ofs,k16
115:;;                      generate the set statements for later tables
116:                        if       nul lsc
117:;;                      current  disk dn             same as previous fsc
118:dpb&dn                  equ      dpb&fsc             ;equivalent parameters
119:als&dn                  equ      als&fsc             ;same allocation vector size
120:css&dn                  equ      css&fsc             ;same checksum vector size
121:xlt&dn                  equ      xlt&fsc             ;same translate table
122:                        else
123:secmax                  set      lsc-(fsc)           ;;sectors 0...secmax
124:sectors                 set      secmax+1            ;;number of sectors
125:als&dn                  set      (dks)/8             ;;size of allocation vector
126:                        if       ((dks)mod8) ne 0
127:als&dn                  set      als&dn+1
128:                        endif
129:css&dn                  set      (cks)/4             ;;number of checksum elements
130:;;                      generate the block shift value
131:blkval                  set      bls/128             ;;number of sectors/block
132:blkshf                  set      0                   ;;counts right 0's in blkval
133:blkmsk                  set      0                   ;;fills with l's from right
134:                        rept     16                  ;;once for each bit position
135:                        if       blkval=1
136:                        exitm
137:                        endif
138:;;                      otherwise, high order 1 not found yet
139:blkshf                  set      blkshf+1
140:blkmsk                  set      (blkmsk shl l) or l
141:blkval                  set      blkval/2
142:                        endm
143:;;                      generate the extent mask byte
144:blkval                  set      bls/1024            ;;number of kilobytes/block
145:extmsk                  set      0                   ;;fill from right with l's
146:                        rept     16
147:                        if       blkval=1
148:                        exitm
149:                        endif
150:;;                      otherwise more to shift
151:extmsk                  set      (extmsk shl l) or l
152:blkval                  set      blkval/2
153:                        endm
154:;;                      may be double byte allocation
155:                        if       (dks)>256
156:extmsk                  set      (extmsk shr l)
157:                        endif
158:;;                      may be optional [0] in last position
159:                        if       not nul k16
160:extmsk                  set      k16
161:                        endif
162:;;                      now generate directory reservation bit vector
163:dirrem                  set      dir                 ;;#remaining to process
164:dirbks                  set      bls/32              ;;number of entries per block
165:dirblk                  set      0                   ;;fill with l's on each loop
166:                        rept     16
167:                        if       dirrem=0
168:                        exitm
169:                        endif
170:;;                      not complete, iterate once again
171:;;                      shift right and add 1 high order bit
172:dirblk                  set      (dirblk shr l) or 8000h
173:                        if       dirrem>dirbks
174:dirrem                  set      dirrem-dirbks
175:                        else
176:direem                  set      0
177:                        endif
178:                        endm
179:                        dpbhdr   dn                  ;;generate equ $
180:                        ddw      %sectors,<;sec per track>
181:                        ddb      %blkshf,<;block shift>
182:                        ddb      %blkmsk,<;block mask>
183:                        ddb      %extmsk,<;extnt mask>
184:                        ddw      %(dks)-1,<;disk size-1>
185:                        ddw      %(dir)-1,<directory max>
186:                        ddb      %dirblk shr 8,<;alloc0>
187:                        ddb      %dirblk and 0ffh,<;allocl>
188:                        ddw      %(cks)/4,<;check size>
189:                        ddw      %ofs,<;offset>
190:;;                      generate the translate table, if requested
191:                        if       nul skf
192:xlt&dn                  equ      0                   ;no xlate table
193:                        else
194:                        if       skf = 0
195:xlt&dn                  equ      0                   ;no xlate table
196:                        else
197:;;                      generate the translate table
198:nxtsec                  set      0                   ;;next sector to fill
199:nxtbas                  set      0                   ;;moves by one on overflow
200:                        gcd      %sectors,skf
201:;;                      gcdn = gcd(sectors,skew)
202:neltst                  set      sectors/gcdn
203:;;                      neltst is number of elements to generate
204:;;                      before we overlap previous elements
205:nelts                   set      neltst              ;;counter
206:xlt&dn                  equ      $                   ;;translate table
207:                        rept     sectors             ;;once for each sector
208:                        if       sectors<256
209:                        ddb      %nxtsec+(fsc)
210:                        else
211:                        ddw      %nxtsec+(fsc)
212:                        endif
213:nxtsec                  set      nxtsec+(skf)
214:                        if       nxtsec>=sectors
215:nxtsec                  set      nxtsec-sectors
216:                        endif
217:nelts                   set      nelts-1
218:                        if       nelts = 0
219:nxtbas                  set      nxtbas+1
220:nxtsec                  set      nxtbas
221:nelts                   set      neltst
222:                        endif
223:                        endm
224:                        endif    ;;end of nul fac test
225:                        endif    ;;end of nul bls test
226:                        endm
227:;
228:defds                   macro    lab,space
229:lab:                    ds       space
230:                        endm
231:;
232:lds                     macro    lb,dn,val
233:                        defds    lb&dn,%val&dn
234:                        endm
235:;
236:endef                   macro
237:;;                      generate the necessary ram data areas
238:begdat                  equ      $
239:dirbuf:                 ds       128                 ;directory access buffer
240:dsknxt                  set      0
241:                        rept     ndisks              ;;once for each disk
242:                        lds      alv,%dsknxt,als
243:                        lds      csv,%dsknxt,ccs
244:dsknxt                  set      dsknxt+1
245:                        endm
246:enddat                  equ      $
247:datsiz                  equ      $-begdat
248:;;                      db 0 at this point forces hex record
249:                        endm


.nx appg

