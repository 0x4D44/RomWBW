.he CP/M Operating System Manual                          H  Glossary
.sp
File Control Block (FCB):
Structure used for accessing files on disk.  Contains the drive, 
filename, filetype, and other information describing a file to be 
accessed or created on the disk.  A file control block consists 
of 36 consecutive bytes specified by the user for file I/O 
functions.  FCB can also refer to a directory element in the 
directory portion of the allocated disk space.  These contain the 
same first 32 bytes of the FCB, but lack the current record and 
random record number bytes.
.sp
.sh
filename:  \c
.qs
Name assigned to a file.  A filename can include a primary 
filename of one to eight characters; a filetype of zero to three characters.
A period separates the primary filename from the filetype.
.sp
.mb 5
.fm 1
.sh
file specification:  \c
.qs
Unique file identifier.  A complete CP/M file specification 
includes a disk drive specification followed by a colon, d:, a 
primary filename of one to eight characters, a period, and a filetype of 
zero to three characters.  For example, b:example.tex is a complete CP/M 
file specification.
.sp
.sh
filetype:  \c
.qs
Extension to a filename.  A filetype can be from zero to three 
characters and must be separated from the primary filename by a 
period.  A filetype can tell something about the file.  Some 
programs require that files to be processed have specific 
filetypes.
.sp
.mb 6
.fm 2
.sp 0
.sh
floppy disk:  \c
.qs
Flexible magnetic disk used to store information.  Floppy disks 
come in 5 1/4- and 8-inch diameters.
.sp
.sh
FSC:  \c
.qs
Parameter in the diskdef macro library specifying the first 
physical sector number.  This parameter is used to determine SPT 
and build XLT.
.sp
.sh
hard disk:  \c
.qs
Rigid, platter-like, magnetic disk sealed in a container.  A hard 
disk stores more information than a floppy disk.
.sp
.sh
hardware:  \c
.qs
Physical components of a computer.
.sp
.sh
hexadecimal notation:  \c
.qs
Notation for base 16 values using the decimal digits and letters 
A, B, C, D, E, and F to represent the 16 digits.  Hexadecimal 
notation is often used to refer to binary numbers.  A binary 
number can be easily expressed as a hexadecimal value by taking 
the bits in groups of 4, starting with the least significant bit, 
and expressing each group as a hexadecimal digit, 0-F.  Thus the 
bit value 1011 becomes 0BH and 10110101 becomes 0B5H.
.sp
.sh
hex file:  \c
.qs
ASCII-printable representation of a command, machine language, 
file.
.sp
.sh
hex file format:  \c
.qs
Absolute output of ASM and MAC for the Intel 8080 is a hex format 
file, containing a sequence of absolute records that give a load 
address and byte values to be stored, starting at the load 
address.
.sp
.sh
HOME:  \c
.qs
BIOS entry point which sets the disk head of the currently 
selected drive to the track zero position.
.sp
.sh
host:  \c
.qs
Physical characteristics of a hard disk drive in a system using 
the blocking and deblocking algorithm.  The term, host, helps 
distinguish physical hardware characteristics from CP/M's logical 
characteristics.  For example, CP/M sectors are always 128 bytes, 
although the host sector size can be a multiple of 128 bytes.
.sp
.sh
input:  \c
.qs
Data going into the computer, usually from an operator typing at 
the terminal or by a program reading from the disk.
.sp
.sh
input/output:  \c
.qs
See \c
.sh
I/O.
.sp
.sh
interface:  \c
.qs
Object that allows two independent systems to communicate with 
each other, as an interface between hardware and software in a 
microcomputer.
.sp
.sh
I/O:  \c
.qs
Abbreviation for input/output.  Usually refers to input/output 
operations or routines handling the input and output of data in 
the computer system.
.sp
.sh
IOBYTE:  \c
.qs
A one-byte field in page zero, currently at location 0003H, that 
can support a logical-to-physical device mapping for I/O.  
However, its implementation in your BIOS is purely optional and 
might or might not be supported in a given CP/M system.  The IOBYTE 
is easily set using the command:
.sp
.ti 8
.nf
STAT <logical device> = <physical device>
.fi
.sp
The CP/M logical devices are CON:, RDR:, PUN:, and LST:; each of 
these can be assigned to one of four physical devices.  The IOBYTE 
can be initialized by the BOOT entry point of the BIOS and 
interpreted by the BIOS I/O entry points CONST, CONIN, CONOUT, 
LIST, PUNCH, and READER.  Depending on the setting of the IOBYTE, 
different I/O drivers can be selected by the BIOS.  For example, 
setting LST:=TTY: might cause LIST output to be directed to a 
serial port, while setting LST:=LPT: causes LIST output to be 
directed to a parallel port.
.sp
.sh
K:  \c
.qs
Abbreviation for kilobyte.  See \c
.sh
kilobyte.
.sp
.sh
keyword:  \c
.qs
See \c
.sh
command keyword.
.sp
.sh
kilobyte (K):  \c
.qs
1024 bytes or 0400H bytes of memory.  This is a standard unit of 
memory.  For example, the Intel 8080 supports up to 64K of memory 
address space or 65,536 bytes.  1024 kilobytes equal one megabyte, 
or over one million bytes.
.sp
.sh
linker:  \c
.qs
Utility program used to combine relocatable object modules into 
an absolute file ready for execution.  For example, LINK-80 \ \ 
creates either a COM or PRL file from relocatable REL files, such 
as those produced by PL/I-80 \ \ .
.sp
.sh
LIST:  \c
.qs
A BIOS entry point to a routine that sends a character to the 
list device, usually a printer.
.sp
.sh
list device:  \c
.qs
Device such as a printer onto which data can be listed or 
printed.
.sp
.sh
LISTST:  \c
.qs
BIOS entry point to a routine that returns the ready status of 
the list device.
.sp
.sh
loader:  \c
.qs
Utility program that brings an absolute program image into memory 
ready for execution under the operating system, or a utility used 
to make such an image.  For example, LOAD prepares an absolute 
COM file from the assembler hex file output that is ready to be 
executed under CP/M.
.sp
.sh
logged in:  \c
.qs
Made known to the operating system, in reference to drives.  A 
drive is logged in when it is selected by the user or an 
executing process.  It remains selected or logged in until you 
change disks in a floppy disk drive or enter CTRL-C at the 
command level, or until a BDOS Function 0 is executed.
.sp
.sh
logical:  \c
.qs
Representation of something that might or might not be the same 
in its actual physical form.  For example, a hard disk can occupy 
one physical drive, yet you can divide the available storage on 
it to appear to the user as if it were in several different 
drives.  These apparent drives are the logical drives.
.sp
.sh
logical sector:  \c
.qs
See \c
.sh
sector.
.sp
.sh
logical-to-physical sector translation table:  \c
.qs
See \c
.sh
XLT.
.sp
.sh
LSC:  \c
.qs
Diskdef macro library parameter specifying the last physical 
sector number.
.sp
.sh
LST:  \c
.qs
Logical CP/M list device, usually a printer.  The CP/M list 
device is an output-only device referenced through the LIST and 
LISTST entry points of the BIOS.  The STAT command allows 
assignment of LST: to one of the physical devices:  TTY:, CRT:, 
LPT:, or UL1:, provided these devices and the IOBYTE are 
implemented in the LIST and LISTST entry points of your CP/M BIOS 
module.  The CP/NET command NETWORK allows assignment of LST: to 
a list device on a network master.  For example, PIP LST:=TEST.SUB
prints the file TEST.SUB on the list device.
.sp
.sh
macro assembler:  \c
.qs
Assembler code translator providing macro processing facilities.  
Macro definitions allow groups of instructions to be stored and 
substituted in the source program as the macro names are 
encountered.  Definitions and invocations can be nested and macro 
parameters can be formed to pass arbitrary strings of text to a 
specific macro for substitution during expansion.
.sp
.sh
megabyte:  \c
.qs
Over one million bytes; 1024 kilobytes.  See \c
.sh
byte, \c
.qs
and \c
.sh
kilobyte.
.sp
.sh
microprocessor:  \c
.qs
Silicon chip that is the central processing unit (CPU) of the 
microcomputer.  The Intel 8080 and the Zilog Z80 are 
microprocessors commonly used in CP/M systems.
.sp
.sh
MOVCPM image:  \c
.qs
Memory image of the CP/M system created by MOVCPM.  This image 
can be saved as a disk file using the SAVE command or placed on 
the system tracks using the SYSGEN command without specifying a 
source drive.  This image varies, depending on the presence of a 
one-sector or two-sector boot.  If the boot is less than 128 
bytes (one sector), the boot begins at 0900H, the CP/M system at 
0980H, and the BIOS at 1F80H.  Otherwise, the boot is at 0900H, 
the CP/M system at 1000H, and the BIOS at 2000H.  In a CP/M 1.4 
system with a one-sector boot, the addresses are the same as for 
the CP/M 2 system--except that the BIOS begins at 1E80H instead 
of 1F80H.
.mb 4
.fm 1
.sp
.sh
MP/M:  \c
.qs
Multi-Programming Monitor control program.  A microcomputer 
operating system supporting multi-terminal access with multi-
programming at each terminal.
.sp
.sh
multi-programming:  \c
.qs
The capability of initiating and executing more than one program 
at a time.  These programs, usually called processes, are time-shared,
each receiving a slice of CPU time on a round-robin 
basis.  See \c
.sh
concurrency.
.sp
.sh
nibble:  \c
.qs
One half of a byte, usually the high-order or low-order 4 bits in 
a byte.
.sp
.sh
OFF:  \c
.qs
Two-byte parameter in the disk parameter block at DPB + 13 bytes.  
This value specifies the number of reserved system tracks.  The 
disk directory begins in the first sector of track OFF.
.sp
.sh
OFS:  \c
.qs
Diskdef macro library parameter specifying the number of reserved 
system tracks.  See \c
.sh
OFF.
.sp
.sh
operating system:  \c
.qs
Collection of programs that supervises the execution of other 
programs and the management of computer resources.  An operating 
system provides an orderly input/output environment between the 
computer and its peripheral devices.  It enables user-written 
programs to execute safely.  An operating system standardizes the 
use of computer resources for the programs running under it.
.mb 6
.fm 2
.sp
.sh
option:  \c
.qs
One of many parameters that can be part of a command tail.  Use 
options to specify additional conditions for a command's 
execution.
.sp
.sh
output:  \c
.qs
Data that is sent to the console, disk, or printer.
.sp
.sh
page:  \c
.qs
256 consecutive bytes in memory beginning on a page boundary, 
whose base address is a multiple of 256 (100H) bytes.  In hex 
notation, pages always begin at an address with a least 
significant byte of zero.
.sp
.sh
page relocatable program:  \c
.qs
See \c
.sh
PRL.
.sp
.sh
page zero:  \c
.qs
Memory region between 0000H and 0100H used to hold critical 
system parameters.  Page zero functions primarily as an interface 
region between user programs and the CP/M BDOS module.  Note that 
in non-standard systems this region is the base page of the 
system and represents the first 256 bytes of memory used by the 
CP/M system and user programs running under it.
.sp
.sh
parameter:  \c
.qs
Value in the command tail that provides additional information 
for the command.  Technically, a parameter is a required element 
of a command.
.sp
.sh
peripheral devices:  \c
.qs
Devices external to the CPU.  For example, terminals, printers, 
and disk drives are common peripheral devices that are not part 
of the processor but are used in conjunction with it.
.sp
.sh
physical:  \c
.qs
Characteristic of computer components, generally hardware, that 
actually exist.  In programs, physical components can be 
represented by logical components.
.sp
.sh
primary filename:  \c
.qs
First 8 characters of a filename.  The primary filename is a 
unique name that helps the user identify the file contents.  A 
primary filename contains one to eight characters and can include any 
letter or number and some special characters.  The primary 
filename follows the optional drive specification and precedes 
the optional filetype.
.sp
.sh
PRL:  \c
.qs
Page relocatable program.  A page relocatable program is stored 
on disk with a PRL filetype.  Page relocatable programs are 
easily relocated to any page boundary and thus are suitable for 
execution in a nonbanked MP/M system.
.sp
.sh
program:  \c
.qs
Series of coded instructions that performs specific tasks when 
executed by a computer.  A program can be written in a
processor-specific language or a high-level language that can be 
implemented on a number of different processors.
.sp
.sh
prompt:  \c
.qs
Any characters displayed on the video screen to help the user 
decide what the next appropriate action is.  A system prompt is a 
special prompt displayed by the operating
system.  The alphabetic character indicates the default drive.  Some 
applications programs have their own special prompts.  See \c
.sh
CP/M prompt.
.qs
.sp
.mb 5
.fm 1
PUN:
Logical CP/M punch device.  The punch device is an output-only 
device accessed through the PUNCH entry point of the BIOS.  In 
certain implementations, PUN: can be a serial device such as a 
modem.
.sp
PUNCH:
BIOS entry point to a routine that sends a character to the punch 
device.
.sp
RDR:
Logical CP/M reader device.  The reader device is an input-only 
device accessed through the READER entry point in the BIOS.
See
PUN:.
.sp
READ:
Entry point in the BIOS to a routine that reads 128 bytes from 
the currently selected drive, track, and sector into the current 
DMA address.
.sp
READER:
Entry point to a routine in the BIOS that reads the next 
character from the currently assigned reader device.
.sp
Read-Only (R/O):
Attribute that can be assigned to a disk file or a disk drive.  
When assigned to a file, the Read-Only attribute allows you to 
read from that file but not write to it.  When assigned to a 
drive, the Read-Only attribute allows you to read any file on the 
disk, but prevents you from adding a new file, erasing or changing 
a file, renaming a file, or writing on the disk.  The STAT 
command can set a file or a drive to Read-Only.  Every file and 
drive is either Read-Only or Read-Write.  The default setting for 
drives and files is Read-Write, but an error in resetting the 
disk or changing media automatically sets the drive to Read-Only 
until the error is corrected.  See also \c
.sh
ROM.
.sp
.sh
Read-Write (R/W):  \c
.qs
Attribute that can be assigned to a disk file or a disk drive.  
The Read-Write attribute allows you to read from and write to a 
specific Read-Write file or to any file on a disk that is in a 
drive set to Read-Write.  A file or drive can be set to either 
Read-Only or Read-Write.
.sp
.sh
record:  \c
.qs
Group of bytes in a file.  A physical record consists of 128 
bytes and is the basic unit of data transfer between the 
operating system and the application program.  A logical record 
might vary in length and is used to represent a unit of 
information.  Two 64-byte employee records can be stored in one 
128-byte physical record.  Records are grouped together to form a 
file.
.sp
.sh
recursive procedure:  \c
.qs
Code that can call itself during execution.
.sp
.mb 6
.fm 2
.sh
reentrant procedure:  \c
.qs
Code that can be called by one process while another is already 
executing it.  Thus, reentrant code can be shared between 
different users.  Reentrant procedures must not be self-
modifying; that is, they must be pure code and not contain data.  
The data for reentrant procedures can be kept in a separate data 
area or placed on the stack.
.sp
.sh
restart (RST):  \c
.qs
One-byte call instruction usually used during interrupt sequences 
and for debugger break pointing.  There are eight restart 
locations, RST 0 through RST 7, whose addresses are given by the 
product of 8 times the restart number.
.sp
.sh
R/O:  \c
.qs
See \c
.sh
Read-Only.
.sp
.sh
ROM:  \c
.qs
Read-Only memory.  This memory can be read but not written and so 
is suitable for code and preinitialized data areas only.
.sp
.sh
RST:  \c
.qs
See \c
.sh
restart.
.sp
.sh
R/W:  \c
.qs
See \c
.sh
Read-Write.
.sp
.sh
sector:  \c
.qs
In a CP/M system, a sector is always 128 consecutive bytes.  A 
sector is the basic unit of data read and written on the disk by 
the BIOS.  A sector can be one 128-byte record in a file or a 
sector of the directory.  The BDOS always requests a logical 
sector number between 0 and (SPT-1).  This is typically 
translated into a physical sector by the BIOS entry point 
SECTRAN.  In some disk subsystems, the disk sector size is larger 
than 128 bytes, usually a power of two, such as 256, 512, 1024, or 
2048 bytes.  These disk sectors are always referred to as host 
sectors in CP/M documentation and should not be confused with 
other references to sectors, in which cases the CP/M 128-byte 
sectors should be assumed.  When the host sector size is larger 
than 128 bytes, host sectors must be buffered in memory and the 
128-byte CP/M sectors must be blocked and deblocked from them.  
This can be done by adding an additional module, the blocking and 
deblocking algorithm, between the BIOS disk I/O routines and the 
actual disk I/O.
.sp
.sh
sectors per track (SPT):  \c
.qs
A two-byte parameter in the disk parameter block at DPB + 0.  The 
BDOS makes calls to the BIOS entry point SECTRAN with logical 
sector numbers ranging between 0 and (SPT - 1) in register BC.
.sp
.sh
SECTRAN:  \c
.qs
Entry point to a routine in the BIOS that performs
logical-to-physical sector translation for the BDOS.
.sp
.sh
SELDSK:  \c
.qs
Entry point to a routine in the BIOS that sets the currently 
selected drive.
.sp
.sh
SETDMA:  \c
.qs
Entry point to a routine in the BIOS that sets the currently 
selected DMA address.  The DMA address is the address of a
128-byte buffer region in memory that is used to transfer data to 
and from the disk in subsequent reads and writes.
.sp
.sh
SETSEC:  \c
.qs
Entry point to a routine in the BIOS that sets the currently 
selected sector.
.sp
.sh
SETTRK:  \c
.qs
Entry point to a routine in the BIOS that sets the currently 
selected track.
.sp
.sh
skew factor:  \c
.qs
Factor that defines the logical-to-physical sector number 
translation in XLT.  Logical sector numbers are used by the BDOS 
and range between 0 and (SPT - 1).  Data is written in 
consecutive logical 128-byte sectors grouped in data blocks.  The 
number of sectors per block is given by BLS/128.  Physical 
sectors on the disk media are also numbered consecutively.  If 
the physical sector size is also 128 bytes, a one-to-one 
relationship exists between logical and physical sectors.  The 
logical-to-physical translation table (XLT) maps this 
relationship, and a skew factor is typically used in generating 
the table entries.  For instance, if the skew factor is 6, XLT 
will be:
.sp
.nf
.in 8
Logical:    0    1    2    3    4    5    6   ...   25
Physical:   1    7   13   19   25    5   11   ...   22
.fi
.in 0
.sp
The skew factor allows time for program processing without 
missing the next sector.  Otherwise, the system must wait for an 
entire disk revolution before reading the next logical sector.  
The skew factor can be varied, depending on hardware speed and 
application processing overhead.  Note that no sector translation 
is done when the physical sectors are larger than 128 bytes, as 
sector deblocking is done in this case.  See also \c
.sh
sector, SKF, \c
.qs
and \c
.sh
XLT.
.sp
.sh
SKF:  \c
.qs
A diskdef macro library parameter specifying the skew factor to 
be used in building XLT.  If SKF is zero, no translation table is 
generated and the XLT byte in the DPH will be 0000H.
.sp
.sh
software:  \c
.qs
Programs that contain machine-readable instructions, as opposed 
to hardware, which is the actual physical components of a 
computer.
.sp
.sh
source file:  \c
.qs
ASCII text file usually created with an editor that is an input 
file to a system program, such as a language translator or text 
formatter.
.sp
.sh
SP:  \c
.qs
Stack pointer.  See \c
.sh
stack.
.bp
.sh
spooling:  \c
.qs
Process of accumulating printer output in a file while the 
printer is busy.  The file is printed when the printer becomes 
free; a program does not have to wait for the slow printing 
process.
.sp
.sh
SPT:  \c
.qs
See \c
.sh
sectors per track.
.sp
.sh
stack:  \c
.qs
Reserved area of memory where the processor saves the return 
address when a call instruction is received.  When a return 
instruction is encountered, the processor restores the current 
address on the stack to the program counter.  Data such as the 
contents of the registers can also be saved on the stack.  The 
push instruction places data on the stack and the pop instruction 
removes it.  An item is pushed onto the stack by decrementing the 
stack pointer (SP) by 2 and writing the item at the SP address.  
In other words, the stack grows downward in memory.
.sp
.sh
syntax:  \c
.qs
Format for entering a given command.
.sp
.sh
SYS:  \c
.qs
See \c
.sh
system attribute.
.sp
.sh
SYSGEN image:  \c
.qs
Memory image of the CP/M system created by SYSGEN when a 
destination drive is not specified.  This is the same as the 
MOVCPM image that can be read by SYSGEN if a source drive is 
not specified.  See \c
.sh
MOVCPM image.
.sp
.sh
system attribute (SYS):  \c
.qs
File attribute.  You can give a file the system attribute by 
using the SYS option in the STAT command or by using the set file 
attributes function, BDOS Function 12.  A file with the SYS 
attribute is not displayed in response to a DIR command.  If you 
give a file with user number 0 the SYS attribute, you can read 
and execute that file from any user number on the same drive.  
Use this feature to make your commonly used programs available 
under any user number.
.sp
system prompt:
Symbol displayed by the operating system indicating that the 
system is ready to receive input.
See prompt and CP/M prompt.
.sp
.sh
system tracks:  \c
.qs
Tracks reserved on the disk for the CP/M system.  The number of 
system tracks is specified by the parameter OFF in the disk 
parameter block (DPB).  The system tracks for a drive always 
precede its data tracks.  The command SYSGEN copies the CP/M 
system from the system tracks to memory, and vice versa.  The 
standard SYSGEN utility copies 26 sectors from track 0 and 26 
sectors from track 1.  When the system tracks contain additional 
sectors or tracks to be copied, a customized SYSGEN must be used.
.sp
.sh
terminal:  \c
.qs
See \c
.sh
console.
.sp
.sh
TPA:  \c
.qs
Transient Program Area.  Area in memory where user programs run 
and store data.  This area is a region of memory beginning at 
0100H and extending to the base of the CP/M system in high 
memory.  The first module of the CP/M system is the CCP, which 
can be overwritten by a user program.  If so, the TPA is extended 
to the base of the CP/M BDOS module.  If the CCP is overwritten, 
the user program must terminate with either a system reset
(Function 0) call or a jump to location zero in page zero.  The 
address of the base of the CP/M BDOS is stored in location 0006H 
in page zero least significant byte first.
.sp
.sh
track:  \c
.qs
Data on the disk media is accessed by combination of track and 
sector numbers.  Tracks form concentric rings on the disk; the 
standard IBM single-density disks have 77 tracks.  Each track 
consists of a fixed number of numbered sectors.  Tracks are 
numbered from zero to one less than the number of tracks on the 
disk.
.sp
.sh
Transient Program Area:  \c
.qs
See \c
.sh
TPA.
.sp
.sh
upward compatible:  \c
.qs
Term meaning that a program created for the previously released 
operating system, or compiler, runs under the newly released 
version of the same operating system.
.sp
.sh
USER:  \c
.qs
Term used in CP/M and MP/M systems to distinguish distinct 
regions of the directory.
.sp
.sh
user number:  \c
.qs
Number assigned to files in the disk directory so that different 
users need only deal with their own files and have their own 
directories, even though they are all working from the same disk.  
In CP/M, files can be divided into 16 user groups.
.sp
.sh
utility:  \c
.qs
Tool.  Program that enables the user to perform certain 
operations, such as copying files, erasing files, and editing 
files.  The utilities are created for the convenience of 
programmers and users.
.sp
.sh
vector:  \c
.qs
Location in memory.  An entry point into the operating system 
used for making system calls or interrupt handling.
.sp
.sh
warm start:  \c
.qs
Program termination by a jump to the warm start vector at 
location 0000H, a system reset (BDOS Function 0), or a CTRL-C 
typed at the keyboard.  A warm start reinitializes the disk 
subsystem and returns control to the CP/M operating system at the 
CCP level.  The warm start vector is simply a jump to the WBOOT 
entry point in the BIOS.
.sp
.sh
WBOOT:  \c
.qs
Entry point to a routine in the BIOS used when a warm start 
occurs.  A warm start is performed when a user program branches 
to location 0000H, when the CPU is reset from the front panel, or 
when the user types CTRL-C.  The CCP and BDOS are reloaded from 
the system tracks of drive A.
.sp
.sh
wildcard characters:  \c
.qs
Special characters that match certain specified items.  In CP/M 
there are two wildcard characters:  ? and *.  The ? can be 
substituted for any single character in a filename, and the * can 
be substituted for the primary filename, the filetype, or both.  
By placing wildcard characters in filenames, the user creates an 
ambiguous filename and can quickly reference one or more files.
.bp
.sh
word:  \c
.qs
16-bit or two-byte value, such as an address value.  Although the 
Intel 8080 is an 8-bit CPU, addresses occupy two bytes and are 
called word values.
.sp
.sh
WRITE:  \c
.qs
Entry point to a routine in the BIOS that writes the record at 
the currently selected DMA address to the currently selected 
drive, track, and sector.
.sp
.sh
XLT:  \c
.qs
Logical-to-physical sector translation table located in the BIOS.  
SECTRAN uses XLT to perform logical-to-physical sector number 
translation.  XLT also refers to the two-byte address in the disk 
parameter header at DPBASE + 0.  If this parameter is zero, no 
sector translation takes place.  Otherwise this parameter is the 
address of the translation table.
.sp
.sh
ZERO PAGE:  \c
.qs
See \c
.sh
page zero.
.qs
.sp 2
.ce
End of Appendix H
.nx appi
